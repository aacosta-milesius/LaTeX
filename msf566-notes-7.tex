\section{High-Frequency Data Analysis}\label{High-Frequency}
\paragraph{Primary Text Reading.} \citeA[chap. 5]{tsay2005aft}\index{Tsay, Ruey}

\subsection{Nonsynchronous Trading}

Market microstructure: Why is it important? 
\begin{enumerate}
\item Important in market design \& operation, e.g. to compare different markets (NYSE vs NASDAQ) 
\item To study price discovery, liquidity, volatility, etc. 
\item To understand costs of trading 
\item Important in learning the consequences of institutional arrangements on observed processes, e.g. 
\begin{itemize}
\item Nonsynchronous trading 
\item Bid-ask bounce 
\item Impact of changes in tick size, after-hour trading, etc. 
\item Impact of daily price limits (many foreign markets) 
\end{itemize}
\end{enumerate}

Nonsynchronous trading: \index{Time Series!differing time periods}
Key implication: may induce serial correlations even when the underlying returns are iid. 

Setup: log returns ${r_t}$ are iid $(\mu, \sigma^2)$ 
For each time index $t$, P(no trade) = $\pi$. 
Cannot observe $r_t$ if there is no trade. 
What is the observed log return series $r^o_t$? 
It turns out $r^o_t$ is given in Tsay Eq. (5.1), 
% TODO: page 11 of lec7-08.pdf

\subsection{Bid-Ask Spread}

\subsection{Transaction Data}

\subsection{Price Change Models}

\subsubsection{Ordered Probit Model}

\subsubsection{Decomposition Model}

\subsection{Duration Models}

\subsubsection{ACD Model}

\subsubsection{Simulation}

\subsubsection{Estimation}

\subsection{Nonlinear Duration Models}

