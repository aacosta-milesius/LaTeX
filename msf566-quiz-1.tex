\documentclass[11pt]{article}
\usepackage{setspace,amssymb,latexsym,amsmath,amscd,epsfig,amsthm}
\oddsidemargin=0in
\evensidemargin=0in
\textwidth=6.3in
\topmargin=-0.5in
%\textheight=9in

\parindent=0in
%\pagestyle{empty}

\input{testpoints}
\begin{document}

\title{MSF 566 - Financial Time Series Analysis \\ Fall 2008 \\ Quiz 1}
\date{}
\maketitle

{Name:} {\underline {\hspace{4.5in}}}
\vspace{2pc}

%%%(modify rules, time, points as appropriate)
%Show all work clearly and in order, and circle your final answers.  Justify your answers algebraically whenever possible; when you do use your calculator, sketch all relevant graphs and write down all relevant mathematics.
You have 30 minutes to take this 20 point quiz.
\vspace{2pc}

% problem
\begin{problem}{5}
\raggedright{Keeping consistency of data in mind, calculate the \emph{mean} and \emph{standard deviation} of this set:}\linebreak
\{20.60, 27.79, 22.91, 22.02, 27.01, 20.07, 64.48, 21.97, 22.25, 22.78, 20.16\}
\vspace{2pc}
\vfill
% SHOULD BE 22.756 and 2.655804 with trimmed data set, NOT 26.54909 and 12.83008
\end{problem}

% problem
\begin{problem}{5}
If we gather end-of-day stock prices every trading day for 3 years, will we have a sample with good external validity to generalize intraday trends? Why or why not?
\vspace{2pc}
\vfill
% OF COURSE NOT. You have nothing on intraday trends.
\end{problem}

% problem
\begin{problem}{5}
Calculate $A \times B$.
\begin{eqnarray}
\mathbf{A} &=& \notag
	\begin{bmatrix}
	12 & 7 & 3 \\
	9 & 1 & 4 \\
	\end{bmatrix} \\
\mathbf{B}&=&  \notag
	\begin{bmatrix}
	3 & 6 \\
	11 & 4 \\
	1 & 6 \\
	\end{bmatrix}
\end{eqnarray}
\vfill
\vfill
% In R:
% A<-matrix(c(12,9,7,1,3,4),2,3)
% B<-matrix(c(3,11,1,6,4,6),3,2)
% A %*% B
%  116   118;     42    82
\end{problem}

\begin{problem}{5}
You have been asked to gather a sample of bond prices and have been told that within the population of 5,000 bonds, 50\% are municipal, 30\% are U.S. Treasury, and 20\% are German bunds. Construct sampling fractions using sample sizes of 80, 40, and 40, respectively. 
\vfill
\vfill
% 3.2%  2.7%  and 4.0%
\end{problem}
\showpoints
\thispagestyle{empty}

\end{document}